\documentclass[12pt]{report}
\usepackage{graphicx}
\usepackage{hyperref}
\bibliographystyle{unsrt}
\begin{document}
\section*{Hotspots: A naive Photonics modules consistency validation}
System is a 2D adapted version of the original particle in cavity
(PIC) architecture~\cite{Huang(2011)}. Particle is a cylinder made of
Ag~\cite{Johnson(1972)} with radius $r_p=0.12(2N+1)$ where $N=200$ is
a useful parameter to defining points per side in the square unit cell
used here for 2D periodic system representation. Thus, cylinder's
diameter is defined by 96/401 points. Cavity has a truncated
cylindrical shape with radius $r_v=0.42(2N+1)$, it is a void in Ag
also with a transverse cut at height $h=1.5r_v$. Host is air. Test was
made by means of calculating $\epsilon^M$ and microscopic electric
field using Photonic modules WE/R2, WE/S, LE/NR2, and LE/S. To compare
results between Longitudinal (LE) and Wave Equation (WE)
homogenization methods, the long wavelength approximation was
introduced for the PIC configuration. We scale unit cell size to
$L=50$ nm, then $r_v\equiv a=21.07$ nm and $r_p\equiv b=6.11$ nm. The
wavenumber is defined by $q=\hbar\omega/1239.8$ and wavevector $\vec
k=1.2q \hat e$, with $\hat e$ the polarization direction. Differences
between results obtained by all of such methods are not meaningful. In
all cases $Nh=200$ Haydock coefficients were used. Fig.\ref{fig:fig1}
displays at the
\begin{figure}[htb]
  \begin{center}
    \includegraphics[width=0.6\textwidth]{fig1}\includegraphics[width=0.7\textwidth]{../figures/000}
    \caption{\label{fig:fig1} Left panel: The real and the imaginary
      parts of the macroscopic dielectric function $\epsilon^M_{yy}$
      component versus photon energy $\hbar\omega$. Points on curve
      indicates $\hbar\omega_0=2.88$. Right panel: Microscopic
      electric field modulus in false color according to scale at the
      right side. Title indicates dimensionless $qL$ and $\vec kL$,
      particle (b) and cavity (a) radius in nm, wave vector direction
      (wk\_dir), and $\hbar\omega_0$. Arrows represents direction of
      the real part of the microscopic field, vertical direction is
      defined by $\hat y$.}
  \end{center}
\end{figure}
left panel, the real and the imaginary parts of $\epsilon^M_{yy}$
component versus photon energy $\hbar\omega$. The right panel of
Fig.\ref{fig:fig1} displays the self consistent microscopic field
modulus in false color with the scale bar at the figure right
side. Arrows represent real part of the microscopic field convenient
scaled and decimated for a clear representations of the field
directions. The particle is just touching the cavity and the indicated
point on the curves at the left part of Fig.\ref{fig:fig1} corresponds
to $\hbar\omega_0=2.88$ eV. The microscopic field are obtained when a
probe field with $\hat e =\hat y$ polarization direction and frequency
$\hbar\omega_0$ excite the system.
\begin{figure}
  \begin{center}
    \includegraphics[width=0.6\textwidth]{fig2}\href{run:totem
      una\_L\_50\_d\_0\_dir\_0.mp4}{\includegraphics[width=0.7\textwidth]{../figures/001}}
    \caption{\label{fig:fig2} Idem as Fig. \ref{fig:fig1}. Now it is
      indicated $\hbar\omega_1=2.63$ eV on the curve at left panel and
      the microscopic field for such frequency at the right
      panel. Note that $qL$ and $kL$ modifies accordingly.}
  \end{center}
\end{figure}
Note that Fig.\ref{fig:fig2} it is similar to Fig.\ref{fig:fig1} but
now for $\hbar\omega_1=2.63$ eV tuning also a resonance with a large
but localizaed microscopic field intensity (hotspot).
\begin{figure}[htb]
  \begin{center}
      \includegraphics[width=0.6\textwidth]{fig3}
    \caption{\label{fig:fig3} The real and the imaginary parts of
      macroscopic dielectric function $\epsilon^M_{xx}$ component
      versus photon energy $\hbar\omega$. Two points are indicated on
      curves at $\hbar\omega_2=2.42$ and $\hbar\omega_3=2.94$}.
  \end{center}
\end{figure}
Fig.\ref{fig:fig3} displays the real and the imaginary part of $\hat
x\hat x$ component of the macroscopic dielectric function.
\begin{figure}[htb]
  \begin{center}
    \href{run:totem una\_d\_0\_dir\_m90.mp4}{%
      \includegraphics[width=0.6\textwidth]{../figures/002}\includegraphics[width=0.6\textwidth]{../figures/003}}
    \caption{\label{fig:fig4}. Image for the microscopic electric
      fields at the two reference points of Fig.\ref{fig:fig3} for
      $\hbar\omega_2=2.42$ (left panel) and $\hbar\omega_3=2.94$
      (right panel). The microscopic electric field video coded in
      {\tt mp4} format can be visualized with a click on figure.}
  \end{center}
\end{figure}
Fig.\ref{fig:fig4} displays the microscopic field for $\hbar\omega_2$
and $\hbar\omega_3$.  The corresponding microscopic electric field for
all frequencies can be visualized via {\em totem} codes or many other
used for to manage {\tt mp4} format.

\bibliography{referencias}
\end{document}
